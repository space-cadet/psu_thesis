% LaTeX Curriculum Vitae Template
%
% Copyright (C) 2004-2009 Jason Blevins <jrblevin@sdf.lonestar.org>
% http://jblevins.org/projects/cv-template/
%
% You may use use this document as a template to create your own CV
% and you may redistribute the source code freely. No attribution is
% required in any resulting documents. I do ask that you please leave
% this notice and the above URL in the source code if you choose to
% redistribute this file.

\documentclass[letterpaper]{article}

\usepackage{hyperref}
\usepackage{geometry}

% Comment the following lines to use the default Computer Modern font
% instead of the Palatino font provided by the mathpazo package.
% Remove the 'osf' bit if you don't like the old style figures.
\usepackage[T1]{fontenc}
\usepackage[sc,osf]{mathpazo}

% Set your name here
\def\name{Deepak Vaid}

% Replace this with a link to your CV if you like, or set it empty
% (as in \def\footerlink{}) to remove the link in the footer:
\def\footerlink{}

% The following metadata will show up in the PDF properties
\hypersetup{
  colorlinks = true,
  urlcolor = blue,
  pdfauthor = {\name},
  pdfkeywords = {physics, gravity, cosmology, quantum field theory},
  pdftitle = {\name: Curriculum Vitae},
  pdfsubject = {Curriculum Vitae},
  pdfpagemode = UseNone
}

\geometry{
  body={6.5in, 8.5in},
  left=1.0in,
  top=1.25in
}

% Customize page headers
\pagestyle{myheadings}
\markright{\name}
\thispagestyle{empty}

% Custom section fonts
\usepackage{sectsty}
\sectionfont{\rmfamily\mdseries\Large}
\subsectionfont{\rmfamily\mdseries\itshape\large}

% Other possible font commands include:
% \ttfamily for teletype,
% \sffamily for sans serif,
% \bfseries for bold,
% \scshape for small caps,
% \normalsize, \large, \Large, \LARGE sizes.

% Don't indent paragraphs.
\setlength\parindent{0em}

% Make lists without bullets
\renewenvironment{itemize}{
  \begin{list}{}{
    \setlength{\leftmargin}{1.5em}
  }
}{
  \end{list}
}

\begin{document}

% Place name at left
{\huge \name}

% Alternatively, print name centered and bold:
%\centerline{\huge \bf \name}

\vspace{0.25in}

\begin{minipage}{0.45\linewidth}
  A-16, 1st Floor (East Facing) \\
  South Extension, Part II \\
  New Delhi - 110049 \\
  India
\end{minipage}
\begin{minipage}{0.45\linewidth}
  \begin{tabular}{ll}
    Phone: & +91-9717275953 (Cell) \\
     &  +91-11-65363116 (Landline)\\
    Email: & \href{mailto:dvaid79@gmail.com}{\tt dvaid79@gmail.com} \\
    Homepage: & \href{http://www.phys.psu.edu/people/display/index.html?person_id=1861}{\tt Link} \\
  \end{tabular}
\end{minipage}


\section*{Personal}

\begin{itemize}
\item Citizen of India
\end{itemize}


\section*{Education}

\begin{itemize}
  \item B.S. Physics, University of Missouri at Rolla\footnote{now known as Missouri University of Science and Technology }, 2000--2003
  \item Ph.D. Physics, Pennsylvania State University, 2003--2011, (defended Nov. 2010, completion Oct. 2011)
\end{itemize}


\section*{Research Experience}

\subsection*{Doctoral Research}
\begin{itemize}
\item \textbf{Independent study}, Advisor: Prof. Abhay Ashtekar, Topic: \emph{Geometric Quantum Mechanics}; Pennsylvania State University, (2004--2005)
\item \textbf{Independent study}, Advisor: Prof. Martin Bojowald; Pennsylvania State University, (2005--2006)
\item \textbf{Thesis research}, Advisor: Prof. Stephon H. S. Alexander, Pennsylvania State University, (2006--2010). Research topics covered include:
\begin{enumerate}
\item Studying the effects of LTB metric on anisotropies of the Cosmic Microwave Background using WMAP3 data
\item Applying Markov Chain Monte Carlo (MCMC) to obtain likelihood for Cosmological model fits to WMAP3 data.
\item Condensation of fermions in a cosmological setting
\item Generating Inflation from Condensates
\item Cosmological Parameter Extraction from WMAP3 data via Bayesian Analysis 
\end{enumerate}
\end{itemize}

\subsection*{Undergraduate Research}
\begin{itemize}
\item University of Missouri at Rolla, 2000--2003
\item Advisor: Prof. Donald Madison
\item Numerical Calculation of Ionization Cross-Sections of Noble Gases 
\end{itemize}

\section*{Publications}

\begin{itemize}
\item Gravity Induced Chiral Condensate Formation and the Cosmological Constant (with. S. Alexander), 2006, \href{http://www.arxiv.org/hep-th/0609066}{arXiv:hep-th/0609066}
\item A fine tuning free resolution of the cosmological constant problem (with S. Alexander), 2007, \href{http://www.arxiv.org/hep-th/0702064}{arXiv:hep-th/0702064}
\item Local Void vs Dark Energy: Confrontation with WMAP and Type Ia Supernovae (with S. Alexander, T. Biswas and A. Notari), 2008, \href{http://www.arxiv.org/abs/0712.0370}{arXiv:abs/0712.0370}
\item Embedding the Bilson-Thompson model in an LQG-like framework, 2010, \href{http://www.arxiv.org/abs/1002.1462}{arXiv:abs/1002.1462}
\item Loop Quantum Gravity for the Bewildered, (with S. Bilson-Thompson), 2011, In Progress
\item Elementary Particles as Gates for Universal Quantum Computation, 2011, In Progress
\item The Quantum Hall Effect and Black Hole Entropy, 2011, In Progress.
\item Anti-DeSitter Condensates as Black Hole Interiors, 2012, In Progress.
\end{itemize}

\section*{Talks, Seminars}
\begin{itemize}
	\item \emph{Elementary Particles as Gates for Universal Quantum Computation}, Mehta Research Institute, Allahabad, India, April 7, 2010
	\item \emph{Elementary Particles as Gates for Universal Quantum Computation}, Center for High Energy Physics, Indian Institute of Science, Bangalore, India, April 21, 2010
	\item \emph{Loop Quantum Gravity for the Bewildered}, Physics Department, University of Adelaide, Adelaide, Australia, August 19, 22 and 29, 2011
\end{itemize}

\section*{Teaching Experience}
\begin{itemize}
	\item \textit{Teaching Assistant}, Pennsylvania State University, 2003--2007, duties included 
	\item \textit{Physics Tutor}, University of Missouri at Rolla, 2000--2002, Duties:
\end{itemize}

\section*{Computational Skills}

\begin{itemize}
	\item \emph{Proficient in:} C++ \& Python programming, \LaTeX, Mathematica
	\item \emph{Platforms:} Microsoft Windows, Apple OS X, Ubuntu Linux
	\item \emph{Familiar with:} Matlab, SAGE
%	\item \emph{Projects:}
%	\begin{enumerate}
%		\item Numerical Computations on Clusters. 
%	\end{enumerate}
\end{itemize}

%\subsection*{Current Projects}
%
%\begin{enumerate}
%\item \textbf{Particles as topological structures in quantum gravity}: In a recent paper (\href{http://www.arxiv.org/abs/1002.1462}{arXiv.org:abs/1002.1462}) I have shown one can embed the Bilson-Thompson braid model of elementary particles in a spin-network like framework.
%\item \textbf{Tetrad Condensation}: The action for General Relativity can be cast into a simple form called the Quadratic Spinor Lagrangian. In this form, it becomes manifest that one can treat the tetrad fields as fermionic matter variables. The presence of a non-zero positive cosmological constant term provides the four-fermionic interaction term necessary for the formation of "cooper pairs" which then play the role of atoms of geometry in a theory of quantum gravity (\emph{work in progress})
%\item \textbf{deSitter Hamiltonian and Spin Systems}: The hamiltonian for gravity with non-zero $\Lambda$ (i.e. the deSitter solution) in the connection formulation can be shown to have the same form as the hamiltonian of a spin-system, when we treat tetrads as spins (\emph{work in progress})
%\item \textbf{The Computational Universe}: There exist deep connections between computational and physical law. Recent work on quantum computation and information theory has only made the question more urgent. In the above mentioned paper (\href{http://www.arxiv.org/abs/1002.1462}{arXiv.org:abs/1002.1462}) it was shown that the braid model of elementary particles can be embedded in a spin-network like framework. This combined with the observation by Lomonaco and Kauffmann (\href{http://www.arxiv.org/quant-ph/0401090}{arXiv:quant-ph/0401090}) that the elements of the braid group $B_3$ form a set of universal gates for quantum computation, lead us to a concrete theoretical model of the \emph{Computational Universe} hypothsis.  (\emph{work in progress})
%\end{enumerate}

\section*{References}

\begin{itemize}
	\item \href{http://www.adelaide.edu.au/directory/sundance.bilson-thompson}{Sundance Bilson-Thompson}, Ramsay Postdoctoral Fellow, School of Chemistry and Physics, University of Adelaide, sundancebt@gmail.com
	\item \href{http://www.phys.psu.edu/~jain/}{Jainendra K. Jain}, Erwin W. Mueller Professor of Physics, Pennsylvania State University, PA, USA, jain@phys.psu.edu
	\item \href{http://www.haverford.edu/faculty/salexand}{Stephon Alexander}, Associate Professor of Physics, Haverford College, Haverford, PA, USA, salexand@haverford.edu
	\item \href{http://www.phys.psu.edu/people/display/index.html?person_id=417}{Martin Bojowald}, Associate Professor of Physics, Pennsylvania State University, PA, USA, bojowald@gravity.psu.edu
\end{itemize}

% Footer
\begin{center}
  \begin{footnotesize}
    Last updated: \today \\
    \href{\footerlink}{\texttt{\footerlink}}
  \end{footnotesize}
\end{center}

\end{document}

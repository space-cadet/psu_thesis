\section{Many-Body Phenomena}

The ubiquity of many-body phenomena in Nature is sometimes obscured by the traditional physicist's practise of understanding systems  by reducing them to the simplest "single-body" form which can then be treated analytically. The power of mathematical methods unleashed after Newton and Leibniz's discovery of the calculus, along with the reductionist mindset prevalent in western science for much of the past 500 years has imbued us with a false sense of confidence in our ability to understand systems in this manner. During the past century a revolution has occurred against this trend, fueled primarily by discoveries in strongly correlated many-body condensed matter systems. Examples of such phenomena are the superfluid and superconducting states for which a complete understanding can be gained only by abandoning this approach built around single-body dynamics and perturbation theory and embracing the idea that \emph{More is Different}\cite{Anderson1972More}

\subsection{\dots in Classical Mechanics}